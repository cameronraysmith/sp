\begin{frame}
\begin{center}
1. Preliminaries from Measure Theory
\end{center}

\begin{itemize}
\item Let $\Omega$ be a set.
\item Let $\cal{F}$ be a $\sigma$-algebra of subsets of $\Omega$.
\item Let $\textbf{P}:\cal{F} \rightarrow$ $[0, 1]$ be a measure on $\Omega$ s.t. $\textbf{P}(\Omega) =1$. 
\end{itemize}

\textbf{Definition:} $(\Omega, \cal{F}, \textbf{P})$ is called a \textbf{probability space}.\\

Let $X : \Omega \rightarrow \mathbb{R}$  be an $\cal{F}$-measurable function.

i.e., $X^{-1}(A) \in {\cal {F}}$ $\forall A \in \cal B$

We say $X$ is a \textbf{random variable}.\\

\textbf{Definition:} If X is integrable (i.e.\ $\int |X|\; d\textbf{P} < \infty$), then define 
$$\textbf{E}(X) = \int X\; d\textbf{P}.$$

An element of $\cal{F}$ is called an \textbf{event}.  

Two events $A$ and $B$ are said to be \textbf{independent} if $\textbf{P}(A \cap B) = \textbf{P}(A) \textbf{P}(B)$.
\end{frame}

\begin{frame}
\begin{center}
2. A Gambler's Dilemma
\end{center}

\textbf{Example:} Toss a fair coin. Observe 1 of 2 outcomes: Heads or Tails.

Construct the following probability space.

\begin{itemize}
\item $\Omega = \{H,T\}$
\item ${\cal F} = \{\phi, H, T, \Omega\}$
\end{itemize}

Define a probability on $\cal{F}$:\\
$\textbf{P}(\phi)=0$, $\textbf{P}(H) = \frac{1}{2} = \textbf{P}(T)$, $\textbf{P}(\Omega)=1$

Define the rv $X$ to be outcome of toss:\\
$X=x$ where $x \in \{H,T\}$

Repeatedly toss this coin.  

Let $\{X_{j}\}$, $j = 0,1,2,\dots$ be a collection of rv's s.t. $\textbf{P}(X_{j}=H) = \frac{1}{2} = \textbf{P}(X_{j}=T)$.

\end{frame}