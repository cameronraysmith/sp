\subsection{Definition}
\begin{frame}
Intuitively, \textbf{stochastic processes} are $\infty$-dimensional probability distributions.
\end{frame}

\begin{frame}
\begin{itemize}
\item In most applications, stochastic processes model systems that evolve randomly in time, which is likely the origin of the word \emph{process} in the name, but stochastic processes are not restricted to the temporal metaphor. Think about \emph{order} as a generalization of time.
\item The order of this evolution can be described through the use of an index and an index set respectively $t \in T$.
\end{itemize}

\newtheorem{stpo}{Definition}
\begin{stpo}
Consider a random experiment with sample space $\Omega$, a $\sigma$-algebra $\Sigma$, a base probability measure $\mu : \Sigma \rightarrow [0,1]$, and a collection of random variables $X_t$ indexed by a set $T$. A \textbf{stochastic process} is then defined by the set $\{X_t, t \in T\}$.
\end{stpo}
\begin{itemize}
\item This definition can be specialized to the case of discrete or continuous stochastic processes by taking the index set to be $T \in \mathbb{N}$ or $T \in \mathbb{R}_+$ respectively.
\end{itemize}
\end{frame}

\begin{frame}
In the discrete case imagine indexing by natural numbers, $\mathbb{N}$, such that the process could be represented as a sequence $\{ X_n, n = 0,1,2,\ldots \}$.
\end{frame}

\begin{frame}
How should we understand the continuous case?
\begin{itemize}
\item For each $\omega \in \Omega$, consider $X_t(\omega)=g_\omega(t)$. $g_\omega(t)$ can then be thought of as a function of $t$ that realizes or samples from the stochastic process.
\item For any given $t$, $X_t$ is a random variable, thus to completely describe the stochastic process we need a description of the joint family of random variables $\{ X_t, t \in T \}$ as opposed to just the individual random variables as if they were independent.
\item For any discrete subset of times $\{ t_1, \ldots, t_n \}$ such that $t_1 < \cdots < t_n$ and associated $\{ x_1, \ldots, x_n \}$ we must determine $	P(X_{t_1} \leq x_1, \ldots, X_{t_n} \leq x_n) $
\end{itemize}
\end{frame}

\begin{frame}

How should we understand the continuous case?
\begin{itemize}
\item The \textbf{Kolmogorov extension theorem} ensures that the potentially infinite distribution $\{ X_t,t \in T \}$ where, for example $T \in \mathbb{R_+}$, i.e. $T = [0, \infty)$, can be generated from the finite-dimensional families defined by $	P(X_{t_1} \leq x_1, \ldots, X_{t_n} \leq x_n) $ \cite{Øksendal2010}
\end{itemize}
The necessary conditions are
\begin{itemize}
\item Exchangeability: for all permutations $\pi$ of $1,\ldots,n$ and $x_1,\ldots,x_n$, $P(X_{t_1} \leq x_1, \ldots, X_{t_n} \leq x_n) = P(X_{t_{\pi(1)}} \leq x_{\pi(1)}, \ldots, X_{t_{\pi(n)}} \leq x_{\pi(n)})$ 
\item Extendability: for all $x_1,\ldots,x_n$ and $t_{n+1},\ldots,t_{n+m}$, $P(X_{t_1} \leq x_1, \ldots, X_{t_n} \leq x_n) = P(X_{t_1} \leq x_1, \ldots, X_{t_n} \leq x_n, X_{t_{n+1}} < \infty, \ldots, X_{t_{n+m}} < \infty )$ 
\end{itemize}
\end{frame}

\begin{frame}
How should we understand the continuous case?
\begin{itemize}
\item Given that the conditions are satisfied then there exists a probability space $(\Omega, \Sigma, \mathbb{P})$ with an associated stochastic process $X_t : T \times \Omega \rightarrow \mathbb{R}^n$ with the families $X_{t_1}, \ldots, X_{t_n}$ as finite-dimensional marginal distributions.
\end{itemize}
\end{frame}

\subsection{Gaussian noise}

\begin{frame}
\frametitle{Gaussian (white) noise}
If random variables at different points in time are mutually independent then the joint distribution is simply the product of all its marginals
$$
p_2(x_1,t_1;x_2,t_2)=p_1(x_1,t_1)p_1(x_2,t_2)
$$
\end{frame}

\begin{frame}
\frametitle{Gaussian (white) noise}
The covariance of this distribution is
\begin{multline}
< (X_{t_1} - \mu_1)(X_{t_2} - \mu_2) > \\
= \int dx_1 dx_2 \,\, (x_1 - \mu_1)(x_2 - \mu_2)p_2(x_1,t_1;x_2,t_2) = 0
\end{multline}
where the mean value at time $t_i$ is 
$$
\mu_i = < X_i > = \int dx \,\, x \, p_1(x,t_i)
$$
\end{frame}

\begin{frame}
\frametitle{Gaussian (white) noise}
If $p_1(x,t)$ is independent of time such that $p_1(x,t) \equiv p_1(x)$ then the associated sequence of random variables is said to be identically and independently distributed (IID) with mean $\mu$ and variance $\sigma^2$
$$
X(t) \sim IID(\mu,\sigma^2)
$$
If $p_1(x)$ is the density of a normal distribution then this stochastic process is referred to as Gaussian noise or white noise. The $X(t)$ are then independently and normally distributed at each point in time, which we denote as
$$
X(t) \sim N(0,\sigma^2)
$$
\end{frame}

\begin{frame}
\frametitle{Gaussian (white) noise}
If we denote Gaussian noise as $\eta(t)$ then $< \eta(t) > = 0$. The covariance function between two timepoints in the discrete and continuous cases respectively are
\begin{eqnarray}
< \eta(t)\eta(t') > &=& \sigma^2 \delta_{tt'},\\
< \eta(t)\eta(t') > &=& \sigma^2 \delta(t-t').
\end{eqnarray}

The Fourier transform of the stationary two-time covariance function for a continuous-time process is
$$
F(\omega) = \int d \tau < \eta(t)\eta(t+\tau) > e^{i \omega \tau} = \sigma^2.
$$
\end{frame}

\subsection{Markov processes}

\begin{frame}
\frametitle{Markov processes}
first order Markov processes are characterized by the so-called Markov condition
$$
p_n(x_n,t_n | x_{n-1},t_{n-1}; \ldots ; x_1,t_1)=p_2(x_n,t_n|x_{n-1},t_{n-1}).
$$

The Markov condition means that the probability for a transition from $x_{n-1}$ to $x_n$ during the time interval from $t_{n-1}$ to $t_n$ is independent of the $x_i$ associated to timepoints prior to $t_{n-1}$.

All the information that is useful for attempting to make predictions is embodied in the present.
\end{frame}

\begin{frame}
\frametitle{Markov processes}
The conditional probability $p_2(x,t|x',t')$ is named the transition probability. A Markov process whose transition probability depends only on the time interval is called homogeneous. If, additionally, $p_1(x,t)$ is independent of $t$, then the process is called stationary.

A Markov process is uniquely determined through $p_1(x,t)$ and $p_2(x_2,t_2|x_1,t_1)$. This is a result of the fact that the Markov property ensures that all joint distributions $p_n$ for $n>2$ can be expressed in terms of $p_1$ and $p_2$ as
\begin{multline}
p_n(x_n,t_n;x_{n-1},t_{n-1};\ldots;x_1,t_1)\\
= p_2(x_n,t_n|x_{n-1},t_{n-1})p_2(x_{n-1},t_{n-1}|x_{n-2},t_{n-2}) \cdots p_1(x_1,t_1).
\end{multline}

\end{frame}

\begin{frame}
\frametitle{Markov processes}
$m^{th}$ order Markov processes are characterized by a generalized form of the first-order Markov condition
\begin{multline}
p_n(x_n,t_n | x_{n-1},t_{n-1}; \ldots ; x_1,t_1)\\
= p_{m+1}(x_n,t_n|x_{n-1},t_{n-1}, \ldots ,x_{n-m},t_{n-m}).
\end{multline}

The $m^{th}$ order Markov condition means that the probability for a transition from $x_{n-1}$ to $x_n$ during the time interval from $t_{n-1}$ to $t_n$ is independent of the $x_i$ associated to timepoints prior to $t_{n-m}$.

All the information that is useful for attempting to make predictions is embodied in a history that is $m$ timesteps long.
\end{frame}

\begin{frame}
\frametitle{Markov processes}
The \emph{hidden Markov process} is defined on a pair of random variables $(I,X)$. $I$ is taken to be a discrete random variable over some subset of natural numbers $\{1,2, \ldots ,k \} \subset \mathbb{N}$. Realizations of $X$ are in $\mathbb{R}^m$. Assuming one exists at all, there is a joint density function $p(i,x;t)$ whose marginals are denoted
\begin{eqnarray}
p(i;t) &=& \int d^m x \,\, p(i,x;t)\\
p(x;t) &=& \Sigma_{i=1}^k p(i,x;t)
\end{eqnarray}
and whose conditional densities are denoted
\begin{eqnarray}
p(x;t|i;t) &=& \frac{p(i,x;t)}{p(i;t)}\\
p(i;t|x;t) &=& \frac{p(i,x;t)}{p(x;t)}
\end{eqnarray}
\end{frame}

\begin{frame}
\frametitle{Markov processes}
The dynamics of the hidden Markov process can be specified under the assumption of a Gaussian first-order Markov process. In this case we can express the transition probability, where $p(x,t|i,i',x';t')$ is a Gaussian distribution and $t > t'$, as
$$
p(i,x;t|i',x';t')=p(x,t|i,i',x';t')p(i;t|i',x';t').
$$
\end{frame}

\begin{frame}
\frametitle{Markov processes: examples}
The Wiener process (also called Brownian motion or random walk) has transition probability

$$
p_2(x,t|x',t')=\frac{1}{\sqrt{2 \pi (t-t')}} e^{-(x-x')^2/2(t-t')},
$$

and $p_1(x,t)$ can be taken to be

$$
p_1(x,t) = \frac{1}{\sqrt{2 \pi t}} e^{-x^2/2t}.
$$

As $t \rightarrow 0$, $p_1(x,t)$ approaches a $\delta$ function and for $t \neq 0$, $p_1(x,t)$ is a Gaussian distribution with mean $0$ and variance $t$. This function is also the solution of the deterministic partial differential heat conduction equation in one dimension.
\end{frame}

\begin{frame}
\frametitle{Markov processes: examples}
The Poisson process (a special case of more general point processes) for the random variable $N(t)$ denoting the number of \emph{events} of some type that have occurred up to time $t$ for the discrete variable $n \subset \mathbb{N}$, $\alpha > 0$, and for $t_2 > t_1$ has transition probability 
$$
p_2(n_2,t_2|n_1,t_1) = 
\begin{cases}
\frac{[\alpha (t_2 - t_1)]^{n_2 - n_1}}{(n_2 - n_1)!} e^{-\alpha (t_2 - t_1)}, & \text{for $n_2 \geq n_1$},\\
0 & \text{for $n_2 < n_1$},
\end{cases}
$$
and probability that $n$ events have taken place in the interval $[0,t]$
$$
p_1(n,t) = \frac{(\alpha t)^n}{n!} e^{-\alpha t}
$$
A trajectory of the Poisson process $n(t)$ consists of unit jumps at random times. 
\end{frame}

\subsection{Master equation}
\begin{frame}
\frametitle{Master equations}

The Chapman-Kolmogorov equation is a statement of the transitivity induced by the Markov property that the transition probability of any Markov process must satisfy 

\begin{eqnarray*}
p_2(x_3,t_3;x_1,t_1) &=& \int dx_2 \,\, p_3(x_3,t_3;x_2,t_2;x_1,t_1)\\
					&=& \int dx_2 \,\, p_2(x_3,t_3|x_2,t_2)p_2(x_2,t_2|x_1,t_1)p_1(x_1,t_1)
\end{eqnarray*}

And the conditional probability is then
$$
p_2(x_3,t_3|x_1,t_1) = \int dx_2 \,\, p_2(x_3,t_3|x_2,t_2)p_2(x_2,t_2|x_1,t_1)
$$
\end{frame}

\begin{frame}
\frametitle{Master equations}
If 
$$
\lim_{t \rightarrow \infty} p_1(x,t) \equiv p_{stat}(x) \neq 0
$$
exists, then $p_{stat}$ is named the \emph{stationary distribution}, which is achieved as t approaches $\infty$. Now, because
$$
p_1(x_3,t_3) = \int dx_2 \,\, p_2(x_3,t_3|x_2,t_2)p_1(x_2,t_2),
$$
then for the stationary distribution
$$
p_{stat}(x) = \int dx_2 \,\, p_2(x,t_3|x_2,t_2)p_{stat}(x_2),
$$
for all $t_2$ and $t_3$.
\end{frame}

\begin{frame}
\frametitle{Master equations}
We may place some conditions on the infinitesimal-time behavior of the transition probabilities in order to use the Chapman-Kolmogorov equation to derive a differential equation specifying the dynamics of the density function characterizing the stochastic process.

Imagine a Markov process with transition probability $p_2(x,t+\tau|x'',t)$. For small (and secretly infinitesimal) values of $\tau$ the form of this transition probability will be taken to be
\begin{multline*}
p_2(x,t+\tau|x'',t)\\
= [1-a(x,t)\tau] \delta(x-x'') + \tau w (x,x'',t) + O(\tau^2)
\end{multline*}
\end{frame}

\begin{frame}
\frametitle{Master equations}
Given that the transition probability $p_2(x,t+\tau|x'',t)$ is assumed to be normalized, which means that the integral over its domain of definition is 1, then the equation 
\begin{multline*}
p_2(x,t+\tau|x'',t)\\
= [1-a(x,t)\tau] \delta(x-x'') + \tau w (x,x'',t) + O(\tau^2)
\end{multline*}
implies that
$$
a(x'',t)=\int dx \,\, w(x,x'',t)
$$
\end{frame}

\begin{frame}
\frametitle{Master equations}
From the Chapman-Kolmogorov equation we can derive
\begin{eqnarray*}
p_2(x,t+\tau|x',t') &=& \int dx'' \,\, p_2(x,t+\tau|x'',t)p_2(x'',t|x',t')\\
&=& [1-a(x,t)\tau]p_2(x,t|x',t')\\
& & +\tau \int dx'' \,\, w(x,x'',t)p_2(x'',t|x',t')\\
& & +O(\tau^2)
\end{eqnarray*}

And the master equation results from taking the limit $\tau \rightarrow \infty$
\begin{eqnarray*}
\frac{\partial}{\partial t}p_2(x,t|x',t') &=& \int dx'' \,\, w(x,x'',t)p_2(x'',t|x',t')\\
&-& \int dx'' \,\, w(x'',x,t)p_2(x,t|x',t')
\end{eqnarray*}
\end{frame}

\begin{frame}
\frametitle{Master equations}
A more intuitive form of the master equation given knowledge of the form of the initial distribution is obtained via multiplication by $p_1(x',t')$ and integration over $x'$ to obtain
\begin{eqnarray*}
\frac{\partial}{\partial t}p_1(x,t) &=& \int dx' \,\, w(x,x',t)p_1(x',t)\\
&-& \int dx' \,\, w(x',x,t)p_1(x,t)
\end{eqnarray*}

If the variable $x$ is taken to be discrete $n \in \mathbb{N}$, then $p_n(t) \equiv p_1(x,t)$ and $w_{nn'}(t) \equiv w(x,x',t)$ results in
$$
\dot{p}_n (t) = \Sigma_{n'} [w_{nn'}(t)p_{n'}(t) - w_{n'n}(t)p_{n}(t)]
$$
This discrete form is perhaps the most intuitive. It says that the change in probability of finding the label $n$ at time $t$ can be described in terms of two components. The first is the number of transitions into (or creations) $n' \rightarrow n$ and the second is the number of transitions from (or annihilations) $n \rightarrow n'$.
\end{frame}

\begin{frame}
\frametitle{Master equations}
$$
\dot{p}_n (t) = \Sigma_{n'} [w_{nn'}(t)p_{n'}(t) - w_{n'n}(t)p_{n}(t)]
$$
This discrete state form is perhaps the most intuitive so far. It says that the change in probability of finding the label $n$ at time $t$ can be described in terms of two components. The first is the number of transitions into (or creations) $n' \rightarrow n$ and the second is the number of transitions from (or annihilations) $n \rightarrow n'$.
\end{frame}

\begin{frame}
The Fokker-Planck Equation

\end{frame}

\begin{frame}
Stochastic differential equations

\end{frame}